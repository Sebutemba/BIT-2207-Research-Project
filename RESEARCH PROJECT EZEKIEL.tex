\documentclass[10pt,a4paper]{report}
\usepackage[utf8]{inputenc}
\usepackage{amsmath}
\usepackage{amsfonts}
\usepackage{amssymb}
\author{KIZITO SEBUTEMBA EZEKIEL\\213012200\\13/U/6970/EVE
 }
\title{QUEUE DELAYS AT TELLER COUNTERS IN KAMPALA BANKS.}
\begin{document}
	\begin{center}
		\maketitle 	TOPIC DESCRIPTION
	\end{center}
	\begin{flushleft}
	The world today is characterized by a series of super growing development specifically in the field of technology. There is therefore an intense and ever increasing competition both within and across all  industries as well as financial institutions. This comes as a result of bank operation managers continuing to face wrenching challenges. It's absurd that they must keep up with this pressure for purposes of survival. This means that banks being a major component of the financial system as well as acting as an intermediary between the surplus and deficit sectors of the economy; they are and will always be at the center of attraction to many of their clients. These, however, want to carry out one transaction or the other through the services offered by these banks.\\
	This explains a common feature of banks in Kampala especially their over-crowded banking halls. An event leading to poor levels of client satisfaction and hence movement from one bank to  another in a quest to seek for better banking services without much delay. It so happens that the modern day automation of bank services such as Online banking, Automated Teller Machine (ATM), Mobile banking, etc which were devised as means of minimizing queue delays have not entirely yielded the desired results. This is due to frequent network system breakdowns as well as inadequate high level professionalism among the literates that could manage the sophisticated system servers “pa se”.\\
	Thus this brings about long queues and hence delays specifically in banks within Kampala; leaving many clients wondering why these banks can not employ more staff to reduce on the lagging time. The commitment to get rid of this vice in the sense that there’s minimal time wastage and also getting rid of the numerous number of hours devoted to work which also determines teller's wages.\\
	However, capacity management is a very important aspect in the design of a system service that optimizes operational costs. Such development comes with trade offs between costs of sustaining a standardized level of service delivery and value that clients are attached to.\\
	This trade off analysis is very important in order to meet the profitability objectives of the stakeholders of the banks especially those in Kampala. And by so doing, queue delays at teller counters in Kampala banks can be reduced or even avoided for good.



\subsection*{CASE STUDY OF TOPIC WITH REFERENCE TO DIFFERENT TYPES OF RESEARCH METHODOLOGY.}


	 \subsubsection*{Descriptive Research  Vs Analytical Research.}
	 
\begin{flushleft}
	
•	Descriptive Research(This seeks to determine what a problem is all about.):-
In this case we focus on what queue delays at teller counters is all about. It is about customers to a bank making long queue facing bank tellers in order to carry transaction such as pay bills, make deposits and also withdrawals.\\

•	Analytical Research(This seeks to establish why a problem is the way it is.):-
The case study here relates to result of increased adoption to safer handling and use of a smart way of keeping money which is entirely about banking. Also increased awareness of masses about the uses of financial systems in the region of Kampala. Lastly, the fact that many people in Kampala a business and office people which explains a daily activity relation with banks. 		
	\end{flushleft}	


 \subsubsection*{Basic Research Vs Applied Research.}
\begin{flushleft}

•	Basic Research(This only focuses on nature of use or application in a given field):- Here we shall only look at the fact that there are indeed delays at teller counters. We won’t go deeper into the details. Thats basically the customers, tellers and the money they transact.\\


•	Applied Research(This focuses on a solution to overcome a problem.):-
Here we shall identify the problem which is basically delays at the teller counters in Kampala. One of the solutions to overcome this issue is to come up with a system that can integrate people’s information registered with the bank  by doing a facial recognition scan, offer a user interface that can categorize individuals say as Pre-process(In transit to make a transaction), In-process(At the counter to make transaction) and Post-process(Transactions to be carried out later or in need of approval). Use of machines such as facial scan machine, thumb and signature recognition device to perfect person-info combination, etc
\end{flushleft}
 


\subsubsection*{ Quantitative Research Vs Qualitative Research.}

\begin{flushleft}

•	Qualitative Research(Here we focus on the nature of the problem at hand.):-\\
With reference to the case study we will depend on things like;\\
➢	Nature of service delivery.\\
➢	Do we have literate or illiterate clients?\\
➢	Type of equipment used to carry out transactions.\\
➢	How customers are handled.\\

\end{flushleft}


•	Quantitative Research(Here we look at the amount/quantity of things that problem is about.):-\\
Such things in this case study are;\\

	The number of tellers working on the clients.\\
	Number of hours offered to each client.\\
	Minimum and maximum transactions.\\
	Number of clients handled per hour.\\
	How often are the tellers monitored?\\
	How is the level of service delivery determined?\\


	\end{flushleft}
\end{document}